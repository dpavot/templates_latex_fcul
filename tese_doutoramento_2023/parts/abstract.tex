
\chapter*{\begin{center}Abstract\end{center}}

\noindent{Successful biomedical Relation Extraction (RE) can provide evidence to researchers about possible unknown associations between entities, advancing our current knowledge about those entities and their inherent processes. Current state-of-the-art solutions to perform biomedical RE are based on deep learning approaches with architectures composed of multiple data representations, such as BERT-derivatives (e.g., BioBERT, PubMedBERT, and SciBERT). However, these fail to leverage external knowledge to boost their performance and tend to rely solely on the training data. The ultimate goal of this project was to develop a top-performance RE system that combines the previous language representations with knowledge retrieved from external sources, such as domain-specific ontologies. The main body of this work showcases three deep learning systems based on distinct architectures and with different approaches to knowledge injection, namely, BiLSTMs, recommendation models, and BERT-based language representations, all integrated with knowledge from biomedical ontologies (e.g., Gene Ontology and Human Phenotype Ontology). These systems overcome the previous state-of-the-art in widely used biomedical RE datasets such as the DDI Corpus (drug-drug interactions) and the BC5CDR Corpus (chemical-induced disease interactions). This thesis also presents a new approach to producing RE datasets, using distant supervised techniques allied with crowdsourcing platforms for validation, resulting in the PGR-crowd Corpus that describes human phenotype-gene relations. The systems and approaches created in this thesis were successfully applied and assessed in several case studies (e.g., workshops, challenges, and other relevant applications), for instance, by being awarded the 7\textsuperscript{th} position in the NASA LitCoin NLP Challenge out of approximately 200 participating teams and contributing to the research effort regarding COVID-19.}

\vspace{0.5cm}

\textbf{Keywords:} Deep Learning, Biomedical Relation Extraction, Text Mining, Knowledge Bases, Ontologies. 

