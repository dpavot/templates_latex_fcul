\chapter*{\begin{center}Resumo Alargado\end{center}}

% 1200-1500 Palavras

\noindent{O volume de informação textual não estruturada atualmente disponível nas várias plataformas ultrapassa em muito a capacidade humana de leitura e compreensão. Esta questão é particularmente relevante quando se trata de investigadores e clínicos que precisam de acompanhar os tópicos específicos do seu domínio. A literatura biomédica é o método padrão que esses indivíduos usam para partilhar as suas descobertas, principalmente em artigos, patentes e outros tipos de relatórios técnicos escritos. Assim, os artigos científicos são a principal fonte aberta de conhecimento de relações entre entidades biomédicas, como fenótipos, genes, doenças e medicamentos/compostos químicos.}

Uma fonte abrangente de literatura biomédica é a plataforma PubMed, que combina mais de 35 milhões de citações, o que dificulta a possibilidade de os investigadores e clínicos estarem cientes de todas as associações e dissociações de entidades. Essa falta de conhecimento muitas vezes leva à repetição de experiências para provar ou refutar hipóteses já estudadas. Além disso, mesmo que a hipótese seja única ou nova, a mesma inferência muitas vezes pode ser obtida a partir do conhecimento sobre experiências realizadas em entidades semelhantes. O processamento dessas informações textuais não estruturadas só é viável usando técnicas de prospeção de texto que visam automatizar a Extração de Relações (ER) biomédicas.

Sistemas de prospeção de texto podem se concentrar numa única tarefa ou numa combinação de tarefas. As tarefas mais abordadas incluem reconhecimento de entidades, normalização ou nomeação de entidades, ER e respostas a perguntas. Ao longo dos anos, os investigadores propuseram diversas técnicas e sistemas para abordar essas tarefas. Desde abordagens baseadas em regras até técnicas de aprendizagem profunda, existem inúmeras e incrivelmente complexas maneiras de lidar com essas tarefas, especialmente ao considerar o domínio biomédico.

A aprendizagem profunda é hoje amplamente utilizada para resolver muitos problemas, como reconhecimento de fala, reconhecimento de objetos visuais e compreensão de linguagem natural. A aprendizagem profunda é um tipo de método de aprendizagem automática que aprende representações de dados usando diferentes níveis de abstração dos dados. Nos últimos anos, houve uma progressão do uso de redes neurais recorrentes e redes neurais convolucionais para resolver ER, para a ampla adoção de modelos baseados em \textit{Transformers}. O modelo BERT teve um enorme impacto no campo, impulsionando a criação de muitos sistemas baseados em BERT voltados especificamente para o domínio biomédico, que são generalizáveis para uma ampla gama de tarefas e superaram o anterior estado-da-arte.

Todos os sistemas mencionados acima são treinados usando documentos de texto não estruturados, como artigos do PubMed. No entanto, temos várias representações estruturadas e semi-estruturadas de conhecimento biomédico disponíveis que esses sistemas não consideram. Uma das fontes mais relevantes de conhecimento biomédico estruturado disponível são as ontologias, como a \textit{Gene Ontology} (GO) e a \textit{Human Phenotype Ontology} (HPO), que são ricas em informações sobre processos e interações biomédicas, combinando milhares de termos e anotações.

Fontes externas de conhecimento podem fornecer informações valiosas para detectar relações entre entidades no texto. Essas bases de conhecimento fornecem não apenas características relevantes sobre as respectivas entidades, mas também a semântica subjacente das relações que estabelecem. Outros exemplos de fontes de conhecimento com milhões de entradas são as ontologias \textit{Chemical Entities of Biological Interest} (ChEBI), \textit{Disease Ontology} (DO) e \textit{Unified Medical Language System} (UMLS). As informações contidas nessas fontes provavelmente não são expressas diretamente nos dados de treino em texto não estruturado, mas podem ser úteis para reforçar uma relação entre entidades mencionadas no texto.

Para avaliar sistemas de ER biomédicas, precisamos de conjuntos de dados anotados de qualidade. No entanto, esses conjuntos de dados são escassos, uma vez que exigem conhecimento especializado no domínio, o que é caro e difícil de encontrar. Alguns exemplos de conjuntos de dados estado-da-arte manualmente anotados no campo de ER biomédicas são o \textit{Drug-Drug Interaction} (DDI) e o \textit{Biocreative Chemical Disease Relation} (BC5CDR). A falta de padrões de referência de qualidade no campo dificulta o desenvolvimento de sistemas de ER biomédicas.

A supervisão distante é uma abordagem que pode ser utilizada para criar conjuntos de dados sem depender do conhecimento especializado humano, o que possibilita a criação de grandes conjuntos de dados anotados com vantagens em termos de tempo e custo. No entanto, técnicas de supervisão distante são difíceis de serem confiáveis, especialmente dada a complexidade do domínio biomédico, e muitas vezes exigem revisão humana, diminuindo a vantagem de automação. Da mesma forma, plataformas de \textit{crowdsourcing} podem auxiliar na criação de grandes conjuntos de dados contando com a sabedoria das pessoas para anotar o texto não estruturado. No entanto, essas plataformas também enfrentam os mesmos problemas de confiabilidade nas anotações produzidas, uma vez que encontrar e garantir conhecimento especializado ainda é um desafio.

Portanto, dois dos principais desafios no campo de ER biomédicas são a falta de conhecimento estruturado e de alta qualidade integrado aos sistemas de ER biomédicas, e a necessidade de corpora de referência de qualidade para validar esses sistemas. O sucesso da tarefa de ER biomédicas depende em grande parte de superar esses desafios e encontrar abordagens específicas para melhorar as soluções existentes. No final, o sucesso da ER biomédicas pode ser usado para explorar novas hipóteses experimentais, fornecendo evidências a investigadores e clínicos sobre possíveis associações desconhecidas entre entidades biomédicas.

Esta tese aborda esses desafios. A hipótese central é que o uso de conhecimento externo para realizar ER biomédicas pode melhorar a diversificação, o número e a qualidade das relações extraídas. Para testar a hipótese, o trabalho de investigação foi dividido em dois objetivos principais: Aprendizagem Profunda com Conhecimento Externo e Avaliação.

Inicialmente esta tese apresenta três sistemas de aprendizagem profunda com injeção de conhecimento externo (Objetivo 1). Estes são o BiOnt, o K-BiOnt e o K-RET.

O BiOnt é um sistema baseado em BiLSTM com duas \textit{layers} de anotações que aprendem informações diferentes sobre as entidades nas relações candidatas. As \textit{layers} de anotações do sistema são empregadas utilizando quatro tipos de ontologias biomédicas, nomeadamente a GO, a HPO, a DO e a ChEBI, relativas a produtos genéticos, fenótipos, doenças e compostos químicos, respetivamente. As duas \textit{layers} de anotações representam a concatenação de ancestrais ontológicos das entidades na relação candidata e os ancestrais comuns entre essas entidades quando se consideram relações entre o mesmo tipo de entidades. O BiOnt superou o anterior estado da arte nos três conjuntos de dados usados para avaliação.

Seguindo o BiOnt, para testar outras possibilidades de injeção de conhecimento e beneficiar das ideias de outras áreas, o K-BiOnt utiliza os princípios das recomendações baseadas em \textit{Knowledge Graph} (KG). O K-BiOnt utilizou múltiplos KGs biomédicos para adicionar características às entidades numa relação candidata. O sistema de aprendizagem profunda combinou o BiOnt e um sistema de recomendação baseado em KGs. Os conceitos de item e utilizador foram utilizados para descrever as entidades numa relação candidata e os KGs ontológicos para adicionar informação (ou seja, características) a cada item. Os resultados mostraram que a combinação das duas abordagens pode ajudar a detetar relações mais raras ou subexpressas.

Por fim, dada a ampla utilização do BERT, o K-RET foi desenvolvido utilizando o BERT como modelo de pré-treino e ajustado para incluir conhecimento diretamente no texto dos dados de treino. O K-RET permite a injeção de conhecimento apenas nas entidades da relação candidata, em entidades contextuais adicionais de interesse, utilizando múltiplas fontes de conhecimento e em entidades com múltiplos \textit{tokens}. O K-RET superou significativamente o desempenho dos sistemas BERT de referência, particularmente na deteção de interações medicamentosas.

Após os três sistemas resumidos acima, este trabalho apresenta uma abordagem para a criação de novos conjuntos de dados de ER biomédicas (Objetivo 2).

A abordagem explora a aplicação de técnicas de supervisão distante para recuperar relações candidatas a partir de texto não estruturado e a sua validação através de diferentes técnicas, como uma plataforma de \textit{crowdsourcing} com configurações pré-definidas. Os resultados demonstraram o benefício de ter até mesmo não especialistas a rever um conjunto de dados de relações fenótipo humano-gene utilizando sistemas de última geração para comparação.

Seguidamente, outras explorações de dados biomédicos neste trabalho são relatadas e incluem a consideração de relações negativas, a sua acessibilidade e distinção, a utilização de técnicas de processamento de linguagem natural biomédica para criar um conjunto de dados de perguntas e respostas a partir de fóruns de perguntas e respostas sobre ciências biomédicas e a utilização de recomendações de literatura relacionada com a COVID-19 num ambiente multilingue.

Neste manuscrito, o corpo principal do trabalho aborda a integração de conhecimento externo em sistemas de ER biomédicas, tornando-os predecessores do que mais recentemente definimos como sistemas multimodais que integram mais de um formato/fonte de dados. Além de integrar conhecimento, esta tese apresenta novas formas de criar conjuntos de dados para ER biomédicas e, por meio de workshops e desafios como casos de estudo, diferentes aplicações biomédicas dos princípios relatados no corpo principal do trabalho. Um exemplo dessas aplicações é a obtenção da 7\textsuperscript{a} posição na NASA LitCoin NLP Competition de cerca de 200 equipas participantes. Todas as contibuições desta tese estão em acesso aberto. 

\vspace{0.5cm}

\textbf{Palavras Chave:} Aprendizagem Profunda, Extração de Relações Biomédicas, Prospeção de Texto, Bases de Conhecimento, Ontologias. 