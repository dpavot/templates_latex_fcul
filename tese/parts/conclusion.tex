\hypertarget{5}{}

\chapter{Conclusion}

\rhead{Conclusion}
\lhead{Chapter 5}

\vspace{-1.6cm}

% Gray Line
\begingroup
\color{gray}
\par\noindent\rule{\textwidth}{0.4pt}
\endgroup



\noindent{The main way we communicate scientific knowledge is through scientific literature. At the current rate of document growth, the only way to process this amount of information is by using computational methods. The information obtained through these methods can lead to a better understanding of biological systems. However, as most learning models require a large amount of training data, applying these learning algorithms to biomedical text mining is often unsuccessful due to the lack of training data in biomedical fields. This work made an important contribution to overcame this issue by creating a large and versatile silver standard corpus, the Phenotype-Gene Relations (PGR) corpus. 

When creating biomedical text mining systems, it is essential to take into account the specific characteristics of biomedical literature. Biological information follows different nomenclatures and levels of complexity. The distantly supervised multi-instance and deep learning modules, developed in this work, were successfully built to accommodate the specificities of human phenotype and gene entities. Thus, this work accomplished the initial objectives (Section \hyperlink{1.2}{1.2}) with highly promising results, fulfilling the initial hypothesis (Section \hyperlink{1.2.1}{1.2.1}).}

Following the growing tendency of systems targeting different biomedical relations, there is an increasing need for more domain-specific corpora, that can only be accomplished by automated corpus creation. The PGR corpus consists of 1712 abstracts, 5676 human phenotype annotations, 13835 gene annotations, and 4283 relations\footnote{Query 1, corresponds to the \textit{10/12/2018} release of PGR}. Using Named-Entity Recognition tools and a distantly supervised approach it was possible to effectively identify and extract human phenotype and gene entities and their relations. These results were partially evaluated by eight curators, obtaining a precision of 87.01\%, with an inter-curator agreement of 87.58\%. The PGR corpus was made publicly available to the research community.\footnote{\url{https://github.com/lasigeBioTM/PGR}}

Automatic biomedical Relation Extraction (RE) still has a long way to go to achieve human-level performance scores. Over recent years, some innovative systems have successively achieved better results by making use of multiple knowledge sources and data representations. These systems not only rely on the training data but make use of different language and entity related features, to create models that identify relations in highly heterogeneous text. Although, even with an optimal combination of features and the ideal features to perform biomedical Relation Extraction (RE) tasks are still far from human level performance. Nevertheless, the results achieved by the distantly supervised multi-instance and deep learning modules developed in this dissertation, were respectively, 73.48\% and 55.00\% in F-measure. These modules were able to detect new gold standard relations that were not present in the reference knowledge base. 

This work showed that the knowledge encoded in biomedical ontologies and gold standard knowledge bases plays a vital part in the development of learning systems, providing semantic and ancestry information for entities, such as genes, proteins, phenotypes, and diseases. Also, it produced one freely available silver standard corpus of human phenotype-gene relations; a high-performance distantly supervised multi-instance learning module that can effectively extract human phenotype-gene relations from text; and one deep learning module with an ontological data representation layer (Section \hyperlink{1.4}{1.4}).

Integrating different knowledge sources instead of relying solely on the training data for creating classification models will allow us not only to find relevant information for a particular problem quicker, but also to validate the results of recent research, and propose new experimental hypotheses. 

This work produced three publications including a book chapter about neural networks, a journal paper describing the ontologies applications to deep learning systems, and a conference paper describing the creation of the PGR corpus.


% ------------------------------> FUTURE WORK

\section{Future Work}

For Chapter \hyperlink{3}{3}, future work can include manually correcting the human phenotype annotations that did not match any HPO identifier, with the potential of expanding the number of human phenotype annotations almost 2-fold and increasing the overall recall. Also, to expand the corpus by identifying more missed gene annotations using pattern matching, which is possible due to the approach being fully automated. Another possibility is the expansion of the test-set for more accurate capture of the variance in the corpus. For example, we can select a subset of annotated documents in which two curators could work to grasp the complexity of manually annotating some of these abstracts. Further, it is possible to use semantic similarity to validate the human phenotype-gene relations. Semantic similarity has been used to compare different types of biomedical entities \citep{SSM} and it is a measure of closeness based on their biological role. For example, if the \textit{BRCA1} gene is semantically similar to the \textit{BRAF} gene and the \textit{BRCA1} has an established relation with the \textit{tumor} phenotype, it could be possible to infer that \textit{BRAF} gene also has a relation with the \textit{tumor} phenotype, even if that is not evident by the training data. Finally, the effect of different NER systems applied to the pipeline should be studied.

Regarding the distantly supervised multi-instance learning module, the parameters of the miSVM package could be optimized using cross-validation on the PGR corpus, and different algorithms implemented (besides the sparse multi-instance learning (sMIL) algorithm \citep{Bunescu:2007:MIL:1273496.1273510}). 

For the deep learning module, it is possible to integrate the ontological information in different ways. For instance, one could consider only the relations between the ancestors with the highest information content (more relevant for the candidate pair they characterize). The information content could be inferred from the probability of each term in each ontology or resorting to an external data-set. Also, the already mentioned semantic similarity measurement could account for non-transitive relations (within the same ontology).

Future work may also consist in outperforming the BioBERT application by using their model along with a data representation layer of biomedical ontologies, given that this work already proved to improve the recall when comparing with an identical model that did not resort to ontological information.

Lastly, combining the techniques developed and presented throughout Chapters \hyperlink{3}{3} and \hyperlink{4}{4}, it would be useful to develop a software tool in which we could annotate documents with human phenotype and gene entities and their relations. More than that, to employ and adapt these techniques to other combinations of biomedical entities to further expand our knowledge about biological systems. 
