\chapter*{\begin{center}Abstract\end{center}}

\noindent{Human phenotype-gene relations are fundamental to fully understand the origin of some phenotypic abnormalities and their associated diseases. Biomedical literature is the most comprehensive source of these relations. Several relation extraction tools have been proposed to identify relations between concepts in highly heterogeneous or unstructured text, namely using distant supervision and deep learning algorithms. However, most of these tools require an annotated corpus, and there is no corpus available annotated with human phenotype-gene relations.}

This work presents the Phenotype-Gene Relations (PGR) corpus, a silver standard corpus of human phenotype and gene annotations and their relations (generated in a fully automated manner), and two relation extraction modules using a distantly supervised multi-instance learning algorithm, and an ontology-based deep learning algorithm.
The PGR corpus consists of 1712 abstracts, 5676 human phenotype annotations, 13835 gene annotations, and 4283 relations. The corpus results were partially evaluated by eight curators, all working in the fields of Biology and Biochemistry, obtaining a precision of 87.01\%, with an inter-curator agreement score of 87.58\%.
Distant supervision (or weak supervision) approaches combine an unlabeled corpus with a knowledge base to identify and extract entities from text, reducing the amount of manual effort necessary. Distantly supervised multi-instance learning takes advantage of distant supervision and a sparse multi-instance learning algorithm to train a relation extraction classifier, using a gold standard knowledge base of human phenotype-gene relations.  
Deep learning relation extraction tools, for biomedical text mining tasks, rarely take advantage of existing domain-specific resources, such as biomedical ontologies. Biomedical ontologies play a fundamental role by providing semantic and ancestry information about an entity. This work used the Human Phenotype Ontology and the Gene Ontology, to represent each candidate pair as the sequence of relations between its ancestors for each ontology.
The PGR test-set was applied to the developed relation extraction modules, obtaining promising results, namely 55.00\% (deep learning module), and 73.48\% (distantly supervised multi-instance learning module) in F-measure. This test-set was also applied to BioBERT, a pre-trained biomedical language representation model for biomedical text mining, obtaining 67.16\% in F-measure.

\vspace{0.5cm}

\textbf{Keywords:} Biomedical Literature, Relation Extraction, Silver Standard Corpus, Distant Supervision, Deep Learning. 

