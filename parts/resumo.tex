\chapter*{\begin{center}Resumo\end{center}}

\noindent{As relações entre fenótipos humanos e genes são fundamentais para entender completamente a origem de algumas abnormalidades fenotípicas e as suas doenças associadas. A literatura biomédica é a fonte mais abrangente dessas relações. Diversas ferramentas de extração de relações têm sido propostas para identificar relações entre conceitos em texto muito heterogéneo ou não estruturado, utilizando algoritmos de supervisão distante e aprendizagem profunda. Porém, a maioria dessas ferramentas requer um \textit{corpus} anotado e não há nenhum \textit{corpus} disponível anotado com relações entre fenótipos humanos e genes.} 

Este trabalho apresenta o \textit{corpus} \textit{Phenotype-Gene Relations} (PGR), um \textit{corpus} padrão-prata de anotações de fenótipos humanos e genes e as suas relações (gerado de forma automática) e dois módulos de extração de relações usando um algoritmo de \textit{distantly supervised multi-instance learning} e um algoritmo de aprendizagem profunda com ontologias biomédicas. 
O \textit{corpus} PGR consiste em 1712 resumos de artigos, 5676 anotações de fenótipos humanos, 13835 anotações de genes e 4283 relações. Os resultados do \textit{corpus} foram parcialmente avaliados por oito curadores, todos investigadores nas áreas de Biologia e Bioquímica, obtendo uma precisão de 87,01\%, com um valor de concordância inter-curadores de 87,58\%. 
As abordagens de supervisão distante (ou supervisão fraca) combinam um \textit{corpus} não anotado com uma base de dados para identificar e extrair entidades do texto, reduzindo a quantidade de esforço necessário para realizar anotações manuais. A \textit{distantly supervised multi-instance learning} aproveita a supervisão distante e um \textit{sparse multi-instance learning algorithm} para treinar um classificador de extração de relações, usando uma base de dados padrão-ouro de relações entre fenótipos humanos e genes. 
As ferramentas de aprendizagem profunda de extração de relações, para tarefas de prospeção de textos biomédicos, raramente tiram proveito dos recursos específicos existentes para cada domínio, como as ontologias biomédicas. As ontologias biomédicas desempenham um papel fundamental, fornecendo informações semânticas e de ancestralidade sobre uma entidade. Este trabalho utilizou a \textit{Human Phenotype Ontology} e a \textit{Gene Ontology}, para representar cada par candidato como a sequência de relações entre os seus ancestrais para cada ontologia. 
O \textit{corpus} de teste PGR foi aplicado aos módulos de extração de relações desenvolvidos, obtendo resultados promissores, nomeadamente 55,00\% (módulo de aprendizagem profunda) e 73,48\% (módulo de \textit{distantly supervised multi-instance learning}) na medida-F. Este \textit{corpus} de teste também foi aplicado ao BioBERT, um modelo de representação de linguagem biomédica pré-treinada para prospeção de texto biomédico, obtendo 67,16\% em medida-F.

\vspace{0.5cm}

\textbf{Palavras Chave:} Literatura Biomédica, Extração de Relações, \textit{Corpus} Padrão-Prata, Supervisão Distante, Aprendizagem Profunda. 
