\chapter*{\begin{center}Resumo Alargado\end{center}}

\noindent{A literatura biomédica é o principal meio que os investigadores utilizam para partilhar as suas descobertas, maioritariamente na forma de artigos, patentes e outros tipos de relatórios escritos. Um investigador interessado num tópico específico precisa de estar atualizado em relação aos trabalhos desenvolvidos sobre esse tópico. No entanto, o volume de informação textual disponível supera amplamente a capacidade de análise de um investigador, mesmo restringindo a um domínio específico. Não só isso, mas a informação textual disponível é geralmente apresentada num formato não estruturado ou altamente heterogéneo. Assim, a recuperação de informação relevante exige não só uma quantidade considerável de esforço manual, mas também é uma tarefa que consome demasiado tempo.}

Os artigos científicos são a principal fonte de conhecimento para entidades biomédicas e as suas relações. Essas entidades incluem fenótipos humanos, genes, proteínas, substâncias químicas, doenças e outras entidades biomédicas inseridas em domínios específicos. Uma fonte abrangente de artigos sobre este tópico é a plataforma PubMed, que combina mais de 29 milhões de citações, fornecendo acesso aos seus metadados. O processamento desse volume de informação só é viável através de soluções de prospeção de texto. 

Os métodos automáticos de Extração de Informação (EI) visam obter informações úteis de grandes conjuntos de dados. As soluções de prospeção de texto usam métodos de EI para processar documentos de texto. Os sistemas de prospeção de texto geralmente incluem tarefas de \textit{Named-Entity Recognition} (NER), \textit{Named-Entity Linking} (NEL) e Extração de Relações (ER). O NER consiste em reconhecer entidades mencionadas no texto, identificando o seu primeiro e último carácter. O NEL consiste em mapear as entidades reconhecidas a entradas numa determinada base de dados. A ER consiste em identificar relações entre as entidades mencionadas num determinado documento. Algumas das relações biomédicas comummente extraídas são as interações proteína-proteína, interações fármaco-fármaco e relações gene-doença.

A ER pode ser executada por diferentes métodos, a saber, por ordem de complexidade: coocorrência, baseados em padrões (criados manual e automaticamente), baseados em regras (criados manualmente e automaticamente) e aprendizagem automática (\textit{feature-based}, \textit{kernel-based}, \textit{multi-instance} (MIL) e \textit{recurrent neural networks} (RNN)). O método de \textit{distantly supervised multi-instance learning} utiliza uma base de dados de relações padrão-ouro do domínio de interesse (supervisão distante) combinada com um \textit{sparse multi-instance learning algorithm} (sMIL) para executar a ER. A supervisão distante pressupõe que qualquer frase que mencione um par de entidades correspondente a uma entrada na base de dados provavelmente descreverá uma relação entre essas entidades. Essas relações candidatas podem ser usadas para treinar um classificador usando o algoritmo sMIL. Mais recentemente, técnicas de aprendizagem profunda, como a RNN, provaram obter excelentes resultados em várias tarefas de Processamento de Linguagem Natural (PNL), entre elas a ER. O sucesso da aprendizagem profunda para a PNL biomédica deve-se em parte ao desenvolvimento de modelos de vectores de palavras como o Word2Vec e, mais recentemente, o ELMo, o BERT, o GPT, o Transformer-XL e o GPT-2. Estes modelos aprendem representações vetoriais de palavras que capturam as relações sintáticas e semânticas de palavras e são conhecidos como \textit{word embeddings}. As \textit{Long Short-Term Memory} (LSTM) RNN constituem uma variante de redes neuronais artificiais apresentadas como uma alternativa às RNN. As redes LSTM lidam com frases mais complexas, sendo por isso mais adequadas à literatura biomédica. Em redes LSTM, é possível integrar fontes externas de conhecimento, como ontologias de domínio específico. As ontologias são formalmente organizadas em formatos legíveis por máquinas, facilitando a sua integração em modelos de extração de relações.

O desafio contemporâneo da análise genética é correlacionar os genes aos seus respetivos fenótipos. Os sistemas existentes que têm flexibilidade para serem aplicados na identificação e extração de relações entre fenótipos humanos e genes, oriundos da literatura biomédica, são escassos e limitados. Os principais desafios que eles enfrentam são a falta de dados anotados; dificuldades na identificação de entidades fenotípicas, que são compostas de múltiplas palavras, o que torna complexo a identificação das fronteiras de cada entidade; e uma escassez de especialistas para realizar a correção das relações identificadas. Todos os problemas acima mencionados geram a necessidade de uma criação automatizada de \textit{corpora} e o desenvolvimento de sistemas de aprendizagem automática que possam lidar com a versatilidade das entidades genéticas e fenotípicas humanas e as suas relações, para melhor identificá-las e extraí-las do texto.

Este trabalho divide-se em três etapas, o \textit{corpus} \textit{Phenotype-Gene Relations} (PGR), um \textit{corpus} padrão-prata de anotações de fenótipos humanos e genes e as suas relações (gerado de forma automática), e dois módulos de extração de relações usando um algoritmo de \textit{distantly supervised multi-instance learning} e um algoritmo de aprendizagem profunda com ontologias biomédicas.

Para realizar a primeira etapa, precisamos de um \textit{pipeline} que realize NER para reconhecer genes e entidades fenotípicas humanas, e ER para extrair e classificar uma relação entre cada fenótipo humano e gene identificado. O primeiro passo é coletar resumos de artigos usando a API do PubMed com palavras-chave definidas manualmente, ou seja, cada nome de cada gene que participa numa relação (presente numa base de dados), \textit{homo sapiens} e \textit{disease}. Em seguida, a etapa NER é realizada usando a ferramenta \textit{Minimal Named-Entity Recognizer} (MER) para extrair menções de genes, e a ferramenta \textit{Identifying Human Phenotypes} (IHP) para extrair menções de fenótipos humanos, a partir dos resumos dos artigos. Por fim, usando uma base de dados de relações padrão-ouro, fornecida pela \textit{Human Phenotype Ontology} (HPO), as relações obtidas pela coocorrência das entidades na mesma frase são marcadas como \textit{Conhecida} ou \textit{Desconhecida}. As relacções marcadas com \textit{Conhecida} são relações presentes na base de dados e as relações marcadas com \textit{Desconhecida} são relações que não estão ainda identificadas ou que não existem.
O \textit{corpus} de teste foi criado selecionando aleatoriamente 260 relações para serem revistas por oito curadores (50 relações cada, com uma sobreposição de 20 relações), todos investigadores nas áreas de Biologia e Bioquímica, obtendo uma precisão de 87,01\%, com um valor de concordância inter-curadores de 87,58\%.

Enquanto na primeira etapa se utiliza uma abordagem de supervisão distante para marcar cada relação extraída como \textit{Conhecida} ou \textit{Desconhecida}, na segunda etapa o \textit{corpus} PGR sem anotações vai ser usado para aplicar a abordagem de \textit{distantly supervised multi-instance learning}. Estas duas abordagens de supervisão distante diferem na forma como são aplicadas, como vai ser possível verificar na descrição das respetivas metodologias. 

Na segunda etapa, o objetivo era usar o \textit{corpus} gerado na primeira etapa combinado com uma base de dados (fornecida pelo HPO), que fornece exemplos para a relação que queríamos extrair, para aplicar \textit{distantly supervised multi-instance learning}. A melhor característica desta abordagem de aprendizagem automática é o facto de ela não requerer as anotações de relações, apenas anotações das entidades, neste caso fenótipos humanos e genes, reduzindo a quantidade de esforço necessário para realizar anotações manuais.

Para a última etapa, o objetivo principal foi combinar algoritmos de RNN (aprendizagem profunda) com ontologias biomédicas para melhorar a identificação das relações entre fenótipos e genes humanos na literatura biomédica. As ontologias como o HPO e a \textit{Gene Ontology} fornecem uma representação confiável dos seus respetivos domínios e podem ser usadas como camadas de representação de dados para extrair relações do texto. O sistema proposto representa cada par candidato como a sequência das relações entre as entidades ancestrais na sua respetiva ontologia e combina os \textit{word embeddings} e a WordNet (uma ontologia genérica da língua inglesa) para produzir um modelo capaz de extrair as relações do texto.

O \textit{corpus} de teste PGR foi aplicado aos módulos de extração de relações desenvolvidos, obtendo resultados promissores, nomeadamente 55,00\% (módulo de aprendizagem profunda) e 73,48\% (módulo de \textit{distantly supervised multi-instance learning}) na medida-F. Este \textit{corpus} de teste também foi aplicado ao BioBERT, um modelo de representação de linguagem biomédica pré-treinada para prospeção de texto biomédico, obtendo 67,16\% em medida-F. 

O uso de diferentes fontes de informação, como dados adicionais, para apoiar a procura automatizada de relações entre conceitos biomédicos contribui para o desenvolvimento de farmacogenómica, triagem de testes clínicos e identificação de reações adversas a medicamentos. A identificação de novas relações pode ajudar a validar os resultados de investigações recentes e até propor novas hipóteses experimentais.