\chapter*{\begin{center}Abstract\end{center}}

\noindent{Successful relation extraction can provide evidence to researchers about possible unknown associations between entities, advancing our current knowledge about those entities and their inherent processes. Multiple relation extraction approaches have been proposed to identify relations between concepts in literature, namely using neural networks algorithms. In deep neural networks, the use of multichannel architectures composed of multiple data representations is leading to state-of-the-art results. The correct combination of data representations can eventually lead us to even higher evaluation scores in relation extraction tasks, without the need of complicated feature engineering. Systems that perform Relation Extraction (RE) use Word2Vec word embeddings to capture the syntactic and semantic information about the words. Lately, efforts regarding new pre-trained language representation models have been proposed with BERT and applied to the biomedical domain with BioBERT. Trained on
large-scale biomedical corpora, these systems (BioBERT and BERT) are designed to pre-train deep bidirectional representations by jointly conditioning on both left and right context in all layers. The ultimate goal of this project is to develop a top-performance RE system that combines the previous language representations with semantics retrieved from external sources of knowledge, such as domain-specific ontologies, graph attention mechanisms, and semantic similarity measures. As case studies, this project will use the RE system to fully understand the origin of some phenotypic abnormalities, their associated genes, related diseases, and proteins; and identify cancer-nutrition interactions, regarding a collaboration with the World Health Organization (WHO). The origin of phenotypic abnormalities and the impact of nutrition on cancer are highly relevant biomedical topics where many of the knowledge still needs to be untangled from literature. Extracting these relations can be used to explore new experimental hypotheses providing evidence to researchers and clinicians about possible unknown associations between biomedical entities and populate gold standard knowledge bases. This project is divided into three main goals, which reflect the creation of a state-of-the-art deep learning RE system using semantics for the biomedical domain. The first steps towards the ultimate goal were already accomplished by creating a baseline system with Word2Vec (based on the BO-LSTM system), implementing the use of ontologies within that system, and improving the Phenotype-Gene Relations (PGR) dataset.}

\vspace{0.5cm}

\textbf{Keywords:} Deep Learning System, Biomedical Relation Extraction, Text Mining, Knowledge Bases. 

